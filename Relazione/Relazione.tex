% Un articolo scritto con LaTeX
\documentclass[10pt,a4paper,usenatbib]{article}
\usepackage[round]{natbib}
\usepackage[T1]{fontenc} % imposta la codifica dei font
\usepackage[utf8]{inputenc} % lettere accentate da tastiera
\usepackage[italian]{babel} % per scrivere in italiano
\usepackage{layaureo} % imposta i margini di pagina
\usepackage{url} % per scrivere gli indirizzi Internet
\usepackage{wrapfig}
\usepackage{booktabs} %per poter usare "\toprule, \midrule, \bottomrule"
\usepackage{amsmath}
\usepackage{amsfonts}
\usepackage{amssymb}
\usepackage{graphicx}
\usepackage{tocloft} 
\usepackage{pdfpages}

\newcommand*{\unit}[1]{\ensuremath{\mathrm{\,#1}}}                              % per poter usare il comando "\unit{}"; 
\newcommand{\fpath}{./Figs/}
\addto{\captionsitalian}{\renewcommand{\abstractname}{Introduzione}}

\begin{document}

\author{Alessandro Liberatore \and Luca Rickler}
\title{Simulazione dello sviluppo di uno sciame elettromagnetico in atmosfera}
\maketitle

%INDICE----------------------------------------------------------------------------------------------------------------------------------%

\pagebreak
\tableofcontents
\pagebreak


%INTRODUZIONE----------------------------------------------------------------------------------------------------------------------------------%

\pagebreak

\begin{abstract}
Scopo del seguente programma è la simulazione, con metodi Monte Carlo, dello sviluppo di uno sciame elettromagnetico in atmosfera. 
\\Tale sciame si può immaginare come prodotto dall'interazione di un raggio cosmico primario con gli elementi componenti l'atmosfera terrestre. Si dà la possibilità all'utente di scegliere l'energia della particella (un raggio gamma) che darà inizio allo sciame in un intervallo di energie tra i $10^2\unit{MeV}$ ed i $10^8\unit{MeV}$. 
\\I risultati di tale simulazione possono essere confrontati con quelli di altre simulazioni presenti allo stato dell'arte. 
\end{abstract}


%TESTO----------------------------------------------------------------------------------------------------------------------------------%

\section{Sciami atmosferici}
Un raggio cosmico primario può generare una cascata di particelle secondarie. Questo sciame di particelle viene comunemente chiamato \textit{Extensive Air Showers} (EAS). In particolare questi sciami sono composti sia da una componente elettromagnetica che da una adronica. Nella presente simulazione, si focalizzerà l'attenzione sulla sola componente elettromagnetica generata da un raggio $\gamma$ considerato come particella primaria (sciami generati da elettroni o positroni sono, dopo poche lunghezze di radiazione, del tutto analoghi a quelli generati da raggi gamma). 

\subsection{Sciami elettromagnetici}
Come detto, si può considerare uno sciame elettromagnetico come prodotto dall'interazione di un elettrone, positrone o raggio $\gamma$ con l'atmosfera terrestre. Date le alte energie in gioco ($\approx10^6\unit{MeV}$), i fenomeni di interazione predominanti per queste particelle a tali energie sono:
\begin{itemize}
\item \textbf{bremsstrahlung} per gli elettroni ed i positroni;
\item \textbf{produzione di coppia} per i raggi gamma.
\end{itemize}
Si supponga, a titolo di esempio, un elettrone come particella primaria. Questo emetterà fotoni (per via della radiazione di bremsstrahlung) che potranno a loro volta creare elettroni e positroni tramite la produzione di coppia. Questi leptoni potranno quindi emettere fotoni per bremsstrahlung e così via. Il processo continua fino al raggiungimento delle energie critiche. Queste sono $88.05\unit{MeV}$ per la bremsstrahlung e $10\unit{MeV}$ per la produzione di coppia. 

L'energia critica per la bremsstrhalung in atmosfera (considerata composta da azoto molecolare, ossigeno molecolare ed argon) si è ricavata effettuando prima una media tra l'energia critica per gli elettroni e per i positroni relativa ad ogni singola specie atomica e, successivamente, una media dei risultati pesata sull'abbondanza dei suddetti elementi in atmosfera ($78\% N_2$, $21\% O_2$ ed $1\% Ar$). Si veda l'appendice \ref{appendix:a} per i calcoli estesi. I dati delle energie critiche sono stati presi da \cite{Olive:2016xmw}. 

Per quanto riguarda l'energia di soglia per la produzione di coppie, si è scelto il valore per il quale questo tipo di processo predomina sugli altri. A tal fine, si è ritenuto, come si può osservare nel grafico in \figurename~\ref{\fpath jpg:AttenuazioneFotoni}, che $10\unit{MeV}$ fosse una buona scelta. 
\begin{figure} [!htb]
\centering
\includegraphics[width=11cm, height=11cm, keepaspectratio]{\fpath AttenuazioneFotoni}
\textit{\caption{Diversi meccanismi di attenuazione di fotoni in un mezzo in funzione dell'energia. }}
\label{\fpath jpg:AttenuazioneFotoni}
\end{figure}

Sono dunque stati considerati, fino al raggiungimento delle rispettive energie critiche, esclusivamente i processi sopra citati per lo sviluppo dello sciame. Nel caso di una particella con energia non sufficiente, è stato considerato un processo di riassorbimento spiegato in seguito nel paragrafo \ref{subsec:riassorb}. 

\subsection{Sviluppo dello sciame}
L'atmosfera è stata suddivisa in strati verticali di spessore pari ad una lunghezza di radiazione $\tilde{x}_0 \unit{[g/cm^2]}$. Nel primo strato viene iniettato il raggio cosmico primario che, se sufficientemente energetico (i.e. $E > E_{soglia}$), interagirà attraverso la produzione di coppia o la radiazione di bremsstrahlung. Il processo è simulato ad intervalli costanti $dh$ (step) pari a $0.1\unit{\tilde{x}_0}$.

Il comportamento delle particelle è simulato in modi diversi per ogni specie.
 
Cominciando dalla radiazione bremsstrahlung, dalla letteratura \citep{PassParticMatterPDG}, si trova che il numero di fotoni prodotti da un elettrone di energia $E_e$ in funzione dello spazio percorso e della loro energia, compresa tra $k_{min}$ e $k_{max}$, risulta essere pari a: 
\begin{equation}
N_\gamma = \frac{d}{\tilde{x}_0}\left[\frac{4}{3}\ln{\frac{k_{max}}{k_{min}}}-\frac{4(k_{max} - k_{min})}{3E_e}+\frac{k_{max}^2 - k_{min}^2}{2E_e^2}\right],
\label{eq:Ngamma}
\end{equation}
valida per distanze percorse $d \ll \tilde{x}_0 \unit{[g/cm^2]}$. Si è quindi ricavato il libero cammino medio degli elettroni come la distanza alla quale il numero di $\gamma$ emessi è pari ad uno:
\begin{equation}
\lambda_{BS} = d = \left\{{\frac{\tilde{x}_0 N_\gamma}{\left[\frac{4}{3}\ln{\frac{k_{max}}{k_{min}}}-\frac{4(k_{max} - k_{min})}{3E_e}+\frac{k_{max}^2 - k_{min}^2}{2E_e^2}\right]}}\right\}_{N_\gamma \rightarrow 1}.
\label{eq:lambdaBS}
\end{equation}

Sono stati considerati come valori di $k_{min}$ e $k_{max}$ rispettivamente $1\unit{keV}$ e l'energia del leptone. Il valore minimo è stato scelto arbitrariamente decidendo di ignorare i raggi $\gamma$ di energia inferiore al keV.

La distanza percorsa da ogni singola particella è stata estratta da una distribuzione esponenziale considerando questo libero cammino medio quale parametro. L'energia emessa in forma di fotone è stata estratta tra $1 \unit{keV}$ e l'energia $E$ dell'elettrone per inportance sampling dalla sezione d'urto in funzione dell'energia $k$ del fotone emesso:
\begin{equation}
\frac{d\sigma}{dk} \simeq \frac{E}{k} + \frac 3 4 \frac{k}{E} - 1.
\label{eq:dSigma}
\end{equation}

Nel caso in cui l'energia emessa sia inferiore alla soglia della produzione di coppie, non viene istanziata una nuova particella, ma l'energia emessa viene contata come persa in quello step.

\bigskip

I $\gamma$ emessi possono generare elettroni e positroni per produzione di coppia se sufficientemente energetici. La distanza di interazione è stata considerata fissa, pari a 
\begin{equation}
x = \frac{7}{9}\tilde{x}_0.
\end{equation}
L'energia del fotone viene equiripartita tra le due particelle prodotte.

Ogni particella creata viene emessa ad un certo angolo azimutale $\phi$ estratto casualmente tra 0 e $2\pi$ e ad uno zenitale $\theta$. Quest'ultimo  è stato calcolato come
\begin{equation}
\langle\theta\rangle = \frac{m_e c^2}{E_e}
\end{equation}
per l'emissione dei $\gamma$di bremsstrahlung e 
\begin{equation}
\langle\theta\rangle = \frac{m_e c^2}{E_\gamma}
\end{equation}
per la produzione di coppie. 

Questi due processi vengono eseguiti a cascata su tutti i prodotti fintanto che questi hanno energia sufficiente o non hanno percorso una distanza maggiore dello spessore dello step attuale.

\subsection{Riassorbimento dello sciame}
\label{subsec:riassorb}
Per quanto riguarda il riassorbimento dello sciame, si è proceduto nuovamente in modo diverso per leptoni e raggi gamma. 

I leptoni, una volta che questi non hanno più energia sufficiente ad emettere fotoni, percorrono una distanza estratta da una distribuzione esponenziale con parametro pari al range elettronico in un mezzo che, da letteratura \citep{RevModPhys.24.28}, risulta pari a: 
\begin{equation}
R_{max} = \begin{cases} 0.412 E_e^{1.265-0.0954 \ln{E_e}}     & 0.01\le E_e \le 2.5\unit{MeV} \\ 
                                          0.530 E_e - 0.106                                   & E_e > 2.5\unit{MeV}. 
                   \end{cases}
\end{equation}

I raggi gamma, invece, vengono subito abbandonati quando la loro energia è sotto la soglia della produzione di coppia.
Per semplicità, non vengono considerati processi quali lo scattering Compton e l'effetto fotoelettrico. 

L'energia delle particelle assorbite in questo step viene quindi raccolta in un grafico, insieme a quella dei fotoni emessi a bassa energia.

\section{Descrizione del codice}
Il codice consta di due classi (\textit{Vector3D}, \textit{Particle}), un header contenente le costanti e due macro (\textit{Simulation.C}, \textit{CompileProject.C}).
 
La classe \textit{Vector3D} è utilizzata per le operazioni di calcolo vettoriale in coordinate cilindriche. Le posizioni e le direzioni di moto delle particelle sono gestite con oggetti di questa classe.

La classe \textit{Particle} descrive le variabili cinematiche delle particelle e contiene i metodi \textit{Divide} e \textit{Propagate}, usati rispettivamente per computare i processi di produzione di particelle e l'assorbimento. Questi ritornano dei valori booleani. Nel caso di \textit{Divide}, questi sono pari a vero se il processo viene eseguito e falso altrimenti. Per \textit{Propagate}, vero corrisponde ad una particella trasportata e falso ad una assorbita.

La macro \textit{Simulation.C} contiene le funzioni \textit{EMShower} e \textit{ElongationFit}, responsabili rispettivamente della simulazione di un singolo sciame elettromagnetico e del plot dell'elongation rate.

Per ogni simulazione si è considerato come raggio cosmico primario un $\gamma$ con energia compresa tra $10^2\unit{MeV}$ e $10^8\unit{MeV}$. 

Le particelle prodotte ad ogni step vengono immagazzinate in un oggetto del tipo vector<Particle>. Questo viene quindi passato in input allo step successivo.

\begin{figure} [h!]
\centering
\includegraphics[width=11cm, height=11cm, keepaspectratio]{\fpath MainCicle}
\textit{\caption{Logica del programma.}}
\label{img:MainCicle}
\end{figure} 

La logica generale della simulazione è mostrata nella figura \ref{img:MainCicle}. Per ogni particella viene eseguito il metodo \textit{Divide} e, nel caso di un esito positivo, si immagazzinano i possibili prodotti nel vector di output. Quindi viene immagazzinata la particella stessa se questa non è un fotone. Nel caso di un esito negativo di \textit{Divide}, viene eseguita la funzione \textit{Propagate}, la quale valuta se l'energia della particella sia sufficiente ad effettuare un processo di produzione di coppia o bremsstrahlung. In caso affermativo, questa viene inserita tra gli output, altrimenti viene eseguito l'assorbimento della stessa. Nel caso in cui la particella debba essere assorbita in uno step successivo, viene immagazzinata tra gli output. 


\section{Compilazione ed esecuzione}
La macro \textit{CompileProject.C} contiene la funzione \textit{CompileSimulation}, che si occupa della compilazione di tutti i file necessari per l'esecuzione della simulazione.

Per l'esecuzione, è sufficiente utilizzare la funzione \textit{EMShower}, specificando l'energia del raggio cosmico primario ed, eventualmente, un seed per il generatore casuale, oppure la funzione \textit{ElongationFit}, specificando l'energia minima e massima e l'incremento percentuale in energia per ogni step.

Il codice è stato testato fino ad energie dell'ordine di $100 \unit{TeV}$ per questioni di potenza di calcolo.

\section{Risultati e confronti con altri modelli}
Sono di seguito riportati i grafici di una simulazione con un raggio cosmico primario di energia $100 \unit{TeV}$. La Figura 3 mostra il numero di particelle per strato atmosferico, mentre la Figura 4 mostra l'energia persa dalla sciame per strato atmosferico. Come si può osservare, l'andamento dei due grafici è lo stesso, come ci si aspetta dalla teoria. 

\begin{figure} [!htb]
\centering
\includegraphics[width=9cm, keepaspectratio]{\fpath N_1e8}
\textit{\caption{Numero di particelle per strato atmosferico da uno sciame con primario di energia $100 \unit{TeV}$.}}
\label{img:N_1e8}
\end{figure} 

\begin{figure} [!htb]
\centering
\includegraphics[width=9cm, keepaspectratio]{\fpath En_1e8}
\textit{\caption{Energia persa per strato atmosferico da uno sciame con primario di energia $100 \unit{TeV}$.}}
\label{img:En_1e8}
\end{figure}


I risultati ottenuti sono stati confrontati con una simulazione effettuata con EGS4, mostrata in Figura 5, per verificarne la bontà. 

\begin{figure} [!htb]
\centering
\includegraphics[width=7cm, keepaspectratio]{\fpath geant}
\textit{\caption{Risultati di una simulazione eseguita con Geant4. Sono mostrati il numero di particelle (punti) e l'energia persa (linea continua) \citep{matthews2005heitler}.}}
\label{img:geant}
\end{figure} 

Come si può osservare, l'andamento del numero di particelle e dell'energia persa dallo sciame per strato atmosferico è in accordo con i risultati di altre simulazioni.

Inoltre si è valutata la posizione del massimo della distribuzione al variare dell'energia del primario. Sono state effettuate 10 simulazioni differenti per ogni valore di energia e sono state calcolate la media e la deviazione standard di ogni campione. La Figura 6 mostra i risultati ottenuti (punti) ed il fit lineare associato (retta rossa). Come ci si aspetta dalla teoria (appendice \ref{appendix:b}), l'andamento risulta essere lineare con un coefficiente angolare prossimo a $2$. L'alto valore di $\chi^2$ riscontrato è dovuto ad una sottostima degli errori. Malgrado ciò, l'andamento è marcatamente lineare. La leggera deviazione dal fit osservata ad alte energie potrebbe essere dovuta ad una perdita di validità dell'approssimazione considerata per calcolare il libero cammino medio degli elettroni (equazione \ref{eq:lambdaBS}).

\begin{figure} [!htb]
\centering
\includegraphics[width=9cm, keepaspectratio]{\fpath Eleongation_200_1e6}
\textit{\caption{Elongation rate per sciami di energia compresa tra $200\unit{MeV}$ e $1\unit{TeV}$. Sono mostrati i dati della simulazione (punti) ed il fit lineare (retta rossa).}}
\label{img:elongation}
\end{figure}

%APPENDICI----------------------------------------------------------------------------------------------------------------------------------%

\pagebreak
\appendix
\section{Appendici}
\subsection{Calcolo dell'energia critica e della lunghezza di radiazione per elettroni e positroni in atmosfera}
\label{appendix:a}
Si consideri un'atmosfera composta da: $78\% N_2$, $21\% O_2$ ed $1\% Ar$. Da \cite{Olive:2016xmw} si hanno i seguenti valori: 
\begin{table}[h!]
\centering
\begin{tabular}{lccc}
\toprule
                      &   $x_0$ [m]   &   $\tilde{x}_0$ [$\unit{g/cm^2}$]   &   $E_c$ [MeV]                  \\                          
\midrule
$N_2$:          &   326              &   37.99                               &  $e^-$: 91.74; $e^+$: 89.71        \\
$O_2$:          &   257              &   34.24                               &  $e^-$: 81.45; $e^+$: 79.62         \\
$Ar$:             &   118              &   19.55                               &  $e^-$: 38.03; $e^+$: 37.06          \\
\bottomrule
\end{tabular}
\end{table}
\\Da questi valori è possibile ricavare una lunghezza di radiazione $\tilde{x}_0$ media per elettroni e positroni in atmosfera: \\
\begin{equation*}
\tilde{x}_0 = \frac{\sum\tilde{x}_0^i p^i}{\sum p^i} \approx 37.02\unit{g/cm^2},
\end{equation*}
dove $p^i$ è la concentrazione percentuale della specie i-esima. Per l'energia critica:
\begin{equation*}
E_c = \frac{\sum \frac 1 2 {\left[\left(E_c^-\right)^i + \left(E_c^+\right)^i\right]} p^i}{\sum p^i} \approx 88.05\unit{MeV},
\end{equation*}
dove $\left(E_c^-\right)^i$ ed $\left(E_c^+\right)^i$ sono le energie critiche per elettroni e positroni per la specie i-esima.


\subsection{Modello di Heitler}
\label{appendix:b}
Lo sviluppo di uno sciame elettromagnetico, basato su due processi fondamentali (produzione di coppia e bremsstrahlung), è comunemente descritto, con le opportune approssimazioni e semplificazioni, dal modello di Heitler. 
\\Uno sciame E.M. è generato da un $\gamma$ o un $e^\pm$ che interagisce con l'atmosfera terrestre. La perdita di energia da parte di un $\gamma$ o un $e^\pm$ in atmosfera è descrivibile da: 
\begin{equation}
-\frac{dE}{d\xi} = \frac{E}{\lambda_T}
\end{equation}
da cui ottengo: 
\begin{equation}
E(\xi) = E_0 e^{- \xi/\lambda_T}
\end{equation}
con $\lambda_T$ dell'aria che vale $\approx 37\unit{g cm^{-2}}$. La quantità di materia attraversata dalla particella prima di perdere metà della propria energia è quindi data da: 
\begin{equation}
\frac{E(\xi)}{E_0} = \frac{1}{2} =  e^{- \xi/\lambda_T}
\end{equation}
da cui ottengo: 
\begin{equation}
\xi = \lambda_T \ln{2}. 
\end{equation}
Ciò significa che un $e^\pm$ emette, per bremsstrahlung, un  $\gamma$ dopo avere percorso una distanza di dimezzamento $d = \lambda_T \ln{2}$. Un  $\gamma$ dopo avere percorso la stessa distanza (in realtà sarebbe $\lambda_p = 9/7 \lambda_T$ ma approssimo $\lambda_p \approx \lambda_T$) dà origine ad una coppia elettrone positrone. 
\\Considero che, ad ogni interazione, l'energia dei $e^\pm$ (e dei  $\gamma$) venga divisa \textit{a metà}. Queste sono tutte approssimazioni fatte per semplificare il modello. In \figurename~\ref{\fpath jpg:ModelloHeitler} è schematizzato lo sviluppo di uno sciame E.M. generato da un  $\gamma$ primario interagente con l'atmosfera. 
\begin{figure} [h!]
\centering
\includegraphics[width=5cm, height=11cm, keepaspectratio]{\fpath ModelloHeitler}
\textit{\caption{Sviluppo in atmosfera di uno sciame elettromagnetico generato da un  $\gamma$ (dove "$n$" indica il numero di lunghezze di radiazione, considerate uguali per $e^\pm$ e  $\gamma$, attraversate dallo sciame). }}
\label{\fpath jpg:ModelloHeitler}
\end{figure}
\\Consideriamo uno sciame iniziato da un  $\gamma$ di energia $E_0$. Dopo $n$ lunghezze di dimezzamento $d$ avremo $2^n$ particelle di energia $E_0/2^n$ l'una. 
\\Dunque lo spazio percorso dopo $n$ lunghezze di dimezzamento $d$ sarà: 
\begin{equation}
X = n d, 
\end{equation}
ma anche che:
\begin{equation}
 d = \lambda_T \ln{2}; 
\end{equation}
quindi otteniamo: 
\begin{equation}
 X = n \lambda_T \ln{2}; 
\label{eqn:profondità}
\end{equation}
che mi rappresenta appunto la profondità a cui mi arriva lo sciame. 
\\Poiché $\ln2 = \log_2{2}/\log_2{e}$, ho che il numero di particelle posso scriverlo come: 
\begin{equation}
 N = 2^n = 2^{X/(\lambda_T \ln2)} =  2^{\frac{X}{\lambda_T} \log_2{e}} = e^{\frac{X}{\lambda_T}}. 
\end{equation}
Il numero di particelle al massimo sviluppo dello sciame ($N_{max}$), considerando che la moltiplicazione termina quando l'energia delle particelle diventa inferiore dell'energia critica, è: 
\begin{equation}
\begin{cases}
N_{max} = 2^{n_{critico}} \\
E_0 = N_{max} E_{critico}^\gamma \\
\end{cases}
\end{equation}
da cui otteniamo: 
\begin{equation}
n_{critico} = \frac{1}{\ln2} \ln{\frac{E_0}{E_{critico}^\gamma}} 
\label{eqn:ncritico}
\end{equation}
e, di conseguenza, si ha che: 
\begin{itemize} 
\item Numero particelle al massimo sviluppo dello sciame: $N_{max} \propto E_0$; 
\item  Profondità di massimo sviluppo: $X_{max} \propto \ln(E_0)$; 
\end{itemize}
dove la proporzionalità per la profondità di massimo sviluppo si trova sostituendo alla~(\ref{eqn:profondità}) il valore di $n_{critico}$ trovato in~(\ref{eqn:ncritico}). 
\\Tuttavia raramente gli esperimenti si trovano esattamente al massimo sviluppo di uno sciame. Trovandosi più in basso del massimo dello sciame, misuro principalmente il numero di elettroni presenti in esso. Si trova che il modello descritto sovrastima il numero di elettroni rispetto quello dei $\gamma$. Si introduce quindi una costante di attenuazione $g=10$: 
\begin{equation}
N_e = \frac{N_{max}}{g}. 
\label{eqn:gNe}
\end{equation}
\\Sperimentalmente si misura anche la così detta \textit{elongation rate} che è definita come la crescita di $X_{max}$ \textit{in funzione di $E_0$}:  
\begin{equation}
\Delta \equiv \frac{dX_{max}}{d\log{E_0}} 
\end{equation}
e poiché, come visto, ho che: 
\begin{equation}
X_{max} = \lambda_T \ln{10} \log{\frac{E_0}{E_{critico}^\gamma}}
\label{eqn:Xmax}
\end{equation}
(dove ho usato: $\log_{10}{\frac{E_0}{E^\gamma}} = \frac{\ln{(E_0/E^\gamma)}}{\ln{10}}$), ottengo: 
\begin{equation}
\begin{split}
\Delta &= \frac{dX_{max}}{d\log{E_0}} \\
&= \frac{d(\lambda_T \ln{10} \log{(E_0/E^\gamma_{critico})})}{d\log{E_0}} \\
&= \frac{d(2.3\lambda_T (\log{E_0} - \log{E^\gamma_{critico})})}{d\log{E_0}} \\
&= 2.3\lambda_T. 
\end{split}
\end{equation}
Osserviamo quindi che il valore dell'\textit{elongation rate} (che ricordiamo essere la variazione di $X_{max}$ \textit{in funzione di $E_0$}) è circa \textit{costante}! Rappresentando quindi $X_{max}$ in funzione del logaritmo in base dieci dell'energia del primario, mi aspetto di ottenere una retta con coefficiente angolare circa pari a 2.3. 


%BIBLIOGRAFIA----------------------------------------------------------------------------------------------------------------------------------%

\bibliographystyle{plainnat}
\bibliography{Biblio}

\end{document}



%/////////////////////////////////////////////////////////////////////////////////////////////////////////////////////////////////////////////////////%


